% Chapter 2

\chapter{شبکه‌های اطلاعات محور}

%شبکه‌های اطلاعات-محور به تنهایی یک مفهوم و یا به بیان بهتر یک نگرش روی شبکه‌اند. در این راستا پروژه‌هایی در حال انجام هستند. این پروژه‌ها عبارتند از:
%\begin{itemize}
%	\item
%	\textbf{\lr{DONA\footnote{\lr{Data Oriented Network Architecture}}}}\cite{dona}: طراحی مجدد نام‌گذاری در اینترنت موجود در سال ۲۰۰۷ به صورت ناوابسته به معماری کنونی\footnote{\lr{Clean-slate}} 
%	
%	\item
%	\textbf{\lr{CCN\footnote{\lr{Content Centric Network}}}}\cite{ndn}: طراحی صورت گرفته در سال ۲۰۰۹ با نگاهی به \lr{IP} که در قالب پروژه‌ی \lr{NDN}\footnote{\lr{Named Data Networking}} در حال توسعه است.
%	
%	\item
%	\textbf{\lr{PSIRP\footnote{\lr{Publish-Subscribe Internet Routing Paradigm}}}}\cite{pursuit}: هم اکنون در قالب پروژه‌ی \lr{PURSUIT} در حال ادامه است.
%	
%	\item
%	.................
%	
%	
%\end{itemize}

برای بررسی و تحلیل روی شبکه‌های اطلاعات-محور در بیشتر موارد از معماری \lr{CCN} استفاده می‌شود. به همین دلیل دو اصطلاح اطلاعات-محور و محتوا-محور می‌توانند به جای یکدیگر استفاده شوند. پیش از ادامه‌، توجه به این نکته ضروری و راهبردی است:
\begin{quote}
	درخواست و پاسخ آن در شبکه‌های محتوا-محور مانند اینترنت امروزی\footnote{\lr{HTTP-GET}} است با این تفاوت که در لایه‌ی شبکه انجام می‌شود.
\end{quote}

در ادامه ............



\section{نام‌گذاری}
نام‌گذاری\footnote{\lr{Naming}} یکی از اصول شبکه‌های اطلاعات-محور است. در این شبکه‌ها همه‌چیز بر پایه‌ی نام محتوا شکل گرفته، به طور مثال ارسال\footnote{\lr{Forwarding}}، مسیریابی\footnote{\lr{Routing}}، نهان‌سازی و امنیت محتواها، با تکیه بر همین ویژگی انجام می‌شود. 

در شبکه‌ی امروزی درخواست هر محتوا با دانستن مکان آن امکان‌پذیر است؛ بدین معنی که درخواست برای مقصد مورد نظر فرستاده می‌شود. مسیریاب‌ها در شبکه تنها مسئول رساندن آن درخواست به \textbf{مقصد مشخص‌شده} می‌باشند. در حالیکه در شبکه‌ّهای محتوا-مبنا که عنصر اصلی هر درخواست نام محتواست، مسیریاب‌ها علاوه بر ارسال درخواست باید در مورد مقصد نیز تصمیم‌گیری کنند(در قسمت ................. روند یک درخواست مورد بررسی قرار می‌گیرد.).

نام‌گذاری در شبکه‌های اطلاعات-محور در بیشتر مواقع به دو صورت انجام می‌شود:

\begin{itemize}
	\item نام‌گذاری سلسله مراتبی\footnote{\lr{Hierarchical}}
	که قابل خواندن توسط انسان بوده و مشابه آدرس‌های استفاده شده در خدمات نام دامنه\footnote{\lr{Domain Name Service (DNS)}} امروزی است. مقیاس‌پذیری\footnote{\lr{Scalability}} بالا از جمله ویژگی‌های این روش است.
	\item نام‌گذاری خود-تصدیقی\footnote{\lr{Self-certifying}}
	که قابل خواندن برای انسان نبوده و با توجه به اینکه در نام بسته از امضا استفاده شده، نیازی به بررسی اصالت صاحب محتوا\footnote{\lr{Authorization}} نیست. برای این منظور می‌توان از درهم‌سازی شده‌ی\footnote{\lr{Hash}} نام بسته در کنار کلید عمومی صاحب آن استفاده کرد.
	
\end{itemize}

\section{"چه" به جای "کجا"}
در شبکه‌ی امروزی دانستن مکان یک محتوا برای درخواست آن ضروری است به طوری‌که بسته‌های ارسالی در شبکه همگی دارای آدرس \lr{IP} مقصد می‌باشند. همین امر چالش‌هایی در زمینه‌ی تحرک\footnote{\lr{Mobility}} در شبکه به‌وجود می‌آورد. در این حالت مسیریاب‌هایدون اطلاع از محتوای درخواستی، تنها قصد رساندن آن به مقصد مشخص‌شده را دارند.

در شبکه‌های محتوا-محور مسیریاب‌ها اطلاعات بیشتری درباره‌ی محتوای درخواست‌شده دارند. بررسی درخواست‌ها در لایه‌ی شبکه، این امکان را به مسیریاب‌ها می‌دهد که از محتوای بسته‌ی درخواستی با‌خبر شوند و در صورت امکان به آن پاسخ داده و یا برای مقصد بسته تصمیم‌گیری کنند(چیزی که در شبکه‌ی امروزی انجام نمی‌شود).

بنابراین در شبکه‌های محتوا-محور مکان محتوا اهمیتی نداشته و خود محتوا اهمیت پیدا می‌کند. همین امر به نهان‌سازی در این شبکه‌ها کمک شایان ذکری می‌کند. چرا که بدون توجه به مکان می‌توان محتواها را نهان کرد.

\section{نهان‌سازی}
اهمیت نهان‌سازی در شبکه‌های محتوا-محور را می‌توان به وضوح در نهان‌سازی فراگیر\footnote{\lr{Pervasive Caching}} مشاهده کرد؛ چرا که تمامی مسیریاب‌ها در نهان‌سازی سهیم خواهند بود. ساده‌ترین روش استفاده شده، نهان‌سازی همه‌چیز در همه‌جا\footnote{\lr{Caching Everything Everywhere (CEE)}} بوده که تمامی مسیریاب‌ها، تمامی محتواها را نهان می‌کنند. در فصل ......... به تفصیل به نهان‌سازی پرداخته خواهد شد.



