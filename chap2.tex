% Chapter 2

\chapter{شبکه‌های اطلاعات محور}
شبکه‌ی پیشنهاد شده‌ی اولیه تنها برای ارتباطات نقطه-به-نقطه\footnote{\lr{Point-to-point}}ی ساده و کاربردهای خاص طراحی شده بود. با فراگیر شدن آن، کاربری‌های جدیدی نیز بدان افزوده شد. صفحات وب و محتواهایی از قبیل صدا و تصویر شبکه را ملزم به تغییر کردند. تا جایی که امروزه خدماتی از جمله خدمات نام دامنه\footnote{\lr{Domain Name Service}}، شبکه‌های نظیر-به-نظیر\footnote{\lr{P2P Networks}}، شبکه‌های توزیع محتوا\footnote{\lr{Content Delivery Networks (CDN)}} و خدمات فراوان دیگری به صورت پوشش\footnote{\lr{Overlay}} روی شبکه استفاده می‌شوند. در عین حال به عضوی جداناپذیر از شبکه تبدیل شده‌اند. برای درک بهتر نیاز به شبکه‌های اطلاعات-محور، توجه به مثال زیر خالی از لطف نیست.

برنامه‌نویسان در صورت نیاز به اعمال تغییرات گسترده در کدی که سال‌های قبل نوشته‌شده، تمایل به بازنویسی کد دارند. چرا که اجزای کد  دراین حالت هماهنگی بیشتری خواهند داشت. همچنین شرکت‌های نرم‌افزاری پس از چند سال استفاده و تغییر یک کد، آن‌را با نگاهی به کد فعلی، بازنویسی می‌کنند. شبکه‌های اطلاعات-محور، برای چنین منظوری ارائه شده‌اند. این شبکه‌ها بازنگری روی شبکه‌ی امروزی برای هماهنگی بیشتر اجزا و کاهش سربارهای موجود است. در این حالت، خدمات گفته‌شده در ساختار اصلی قرار می‌گیرند.

%شبکه‌های اطلاعات-محور به تنهایی یک مفهوم و یا به بیان بهتر یک نگرش روی شبکه‌اند. در این راستا پروژه‌هایی در حال انجام هستند. این پروژه‌ها عبارتند از:
%\begin{itemize}
%	\item
%	\textbf{\lr{DONA\footnote{\lr{Data Oriented Network Architecture}}}}\cite{dona}: طراحی مجدد نام‌گذاری در اینترنت موجود در سال ۲۰۰۷ به صورت ناوابسته به معماری کنونی\footnote{\lr{Clean-slate}} 
%	
%	\item
%	\textbf{\lr{CCN\footnote{\lr{Content Centric Network}}}}\cite{ndn}: طراحی صورت گرفته در سال ۲۰۰۹ با نگاهی به \lr{IP} که در قالب پروژه‌ی \lr{NDN}\footnote{\lr{Named Data Networking}} در حال توسعه است.
%	
%	\item
%	\textbf{\lr{PSIRP\footnote{\lr{Publish-Subscribe Internet Routing Paradigm}}}}\cite{pursuit}: هم اکنون در قالب پروژه‌ی \lr{PURSUIT} در حال ادامه است.
%	
%	\item
%	.................
%	
%	
%\end{itemize}

برای بررسی و تحلیل روی شبکه‌های اطلاعات-محور در بیشتر موارد از معماری \lr{CCN} استفاده می‌شود. به همین دلیل دو اصطلاح اطلاعات-محور و محتوا-محور می‌توانند به جای یکدیگر استفاده شوند. در ادامه ویژگی‌های این شبکه‌ها بیشتر مورد بررسی قرار می‌گیرد. اما پیش از ادامه‌، توجه به این نکته ضروری و راهبردی است:
\begin{quote}
	درخواست و پاسخ آن در شبکه‌های محتوا-محور مانند اینترنت امروزی\footnote{\lr{HTTP-GET}} است با این تفاوت که در لایه‌ی شبکه انجام می‌شود.
\end{quote}

\section{نام‌گذاری}
نام‌گذاری\footnote{\lr{Naming}} یکی از اصول شبکه‌های اطلاعات-محور است. در این شبکه‌ها همه‌چیز بر پایه‌ی نام محتوا شکل گرفته، به طور مثال ارسال\footnote{\lr{Forwarding}}، مسیریابی\footnote{\lr{Routing}}، نهان‌سازی و امنیت محتواها، با تکیه همین ویژگی انجام می‌شود. نام‌گذاری هر محتوا باید 


\section{نهان‌سازی}
اهمیت نهان‌سازی در شبکه‌های محتوا-محور را می‌توان به وضوح در نهان‌سازی فراگیر\footnote{\lr{Pervasive Caching}} مشاهده کرد؛ چرا که تمامی مسیریاب‌ها در نهان‌سازی سهیم خواهند بود. ساده‌ترین روش استفاده شده، نهان‌سازی همه‌چیز در همه‌جا\footnote{\lr{Caching Everything Everywhere}} بوده که تمامی مسیریاب‌ها، تمامی محتواها را نهان می‌کنند. در فصل ......... به تفصیل به نهان‌سازی پرداخته خواهد شد.



