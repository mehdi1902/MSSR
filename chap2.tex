% Chapter 2

\chapter{شبکه‌های اطلاعات محور}

برای بررسی و تحلیل روی شبکه‌های اطلاعات-محور در بیشتر موارد از معماری \lr{CCN} استفاده می‌شود. به همین دلیل دو اصطلاح اطلاعات-محور و محتوا-محور می‌توانند به جای یکدیگر استفاده شوند. پیش از ادامه‌، توجه به این نکته ضروری و راهبردی است:
\begin{quote}
	درخواست و پاسخ آن در شبکه‌های محتوا-محور مانند اینترنت امروزی\footnote{\lr{HTTP-GET}} است با این تفاوت که در لایه‌ی شبکه انجام می‌شود.
\end{quote}

شبکه‌های اطلاعات محور در واقع ادامه‌ی مدل انتشار/عضویت(pub/sub)\footnote{\lr{Publish/Subscribe}} می‌باشند. همان‌گونه که از اسم این مدل به نظر می‌رسد،دارای دو اصل پایه‌ی انتشار و عضویت است. سرویس‌دهنده با متد انتشار وجود محتوای خود را در شبکه منتشر کرده و هر کاربر برای دستیابی به آن از عضویت استفاده می‌کند. در شبکه‌های اطلاعات محور نیز این ویژگی با نام‌های دیگری به چشم می‌خورد. به طور مثال در معماری \lr{CCN} از \lr{INTEREST} و \lr{REGISTER} استفاده می‌شود \cite{ndn}.

در ادامه ............


\section{ویژگی‌ها}
شبکه‌های اطلاعات-محور دارای ویژگی‌هایی است که در شبکه‌ی امروزی دیده نمی‌شود ویا به صورت‌های دیگری در حال استفاده است. در ادامه برخی از این ویژگی‌ها به صورت مختصر بیان خواهد شد.

\subsection{نام‌گذاری}
نام‌گذاری\footnote{\lr{Naming}} یکی از اصول شبکه‌های اطلاعات-محور است. در این شبکه‌ها همه‌چیز بر پایه‌ی نام محتوا شکل گرفته، به طور مثال ارسال\footnote{\lr{Forwarding}}، مسیریابی\footnote{\lr{Routing}}، نهان‌سازی و امنیت محتواها، با تکیه بر همین ویژگی انجام می‌شود. 

در شبکه‌ی امروزی درخواست هر محتوا با دانستن مکان آن امکان‌پذیر است؛ بدین معنی که درخواست برای مقصد مورد نظر فرستاده می‌شود. مسیریاب‌ها در شبکه تنها مسئول رساندن آن درخواست به مقصد مشخص‌شده می‌باشند. در حالیکه در شبکه‌های محتوا-مبنا که عنصر اصلی هر درخواست نام محتواست، مسیریاب‌ها علاوه بر ارسال درخواست باید در مورد مقصد نیز تصمیم‌گیری کنند(در قسمت ................. روند یک درخواست مورد بررسی قرار می‌گیرد).

نام‌گذاری در شبکه‌های اطلاعات-محور در بیشتر مواقع به دو صورت انجام می‌شود:

\begin{itemize}
	\item نام‌گذاری سلسله مراتبی\footnote{\lr{Hierarchical}}
	که قابل خواندن توسط انسان بوده و مشابه آدرس‌های استفاده شده در خدمات نام دامنه\footnote{\lr{Domain Name Service (DNS)}} امروزی است. مقیاس‌پذیری\footnote{\lr{Scalability}} بالا از جمله ویژگی‌های این روش است.
	\item نام‌گذاری خود-تصدیقی\footnote{\lr{Self-certifying}}
	که قابل خواندن برای انسان نبوده و با توجه به اینکه در نام بسته از امضا استفاده شده، نیازی به بررسی اصالت صاحب محتوا\footnote{\lr{Authorization}} نیست. برای این منظور می‌توان از درهم‌سازی شده‌ی\footnote{\lr{Hash}} نام بسته در کنار کلید عمومی صاحب آن استفاده کرد. این کار هم به امنیت بسته و هم یکتا بودن نام آن کمک می‌کند.
	
\end{itemize}


\subsection{نهان‌سازی}
اهمیت نهان‌سازی در شبکه‌های محتوا-محور را می‌توان به وضوح در نهان‌سازی فراگیر\footnote{\lr{Pervasive Caching}} مشاهده کرد؛ چرا که تمامی مسیریاب‌ها در نهان‌سازی سهیم خواهند بود. ساده‌ترین روش استفاده شده، نهان‌سازی همه‌چیز در همه‌جا\footnote{\lr{Caching Everything Everywhere (CEE)}} بوده که تمامی مسیریاب‌ها، تمامی محتواها را نهان می‌کنند. در معماری‌های مختلف اطلاعات-محور از روش‌های متفاوتی برای نهان‌سازی استفاده می‌شود اما به طور کلی همگی از نهان‌سازی درون‌شبکه‌ای\footnote{\lr{In-network Caching}} استفاده می‌کنند.

با توجه به نهان‌سازی، پاسخ دادن به یک درخواست شامل چندین مرحله می‌شود. نخست مسیریاب در صورت داشتن محتوای درخواستی در نهان‌گاه\footnote{\lr{Cache}} خود، محتوای مورد نظر را برای کاربر ارسال و در غیر این صورت محتوا را از مسیریاب‌های همسایه درخواست می‌کند.

 در فصل ......... به تفصیل به نهان‌سازی پرداخته خواهد شد.

\subsection{مسیریابی}
روش‌های مسیریابی موجود در شبکه‌های امروزی در طول سالیان طولانی پیشرفت زیادی کرده و در عین حال نیازهای جدید را نیز پاسخ می‌گویند. به همین دلیل استفاده از این روش‌ها در شبکه‌های اطلاعات-محور دور از ذهن نیست. هرچند اعمال برخی تغییرات(مثل استفاده از نام به جای آدرس) بدیهی به‌نظر می‌رسد. به طور کلی درباره مسیریابی در این شبکه‌ها می‌توان گفت که هر روش مسیر‌یابی که در شبکه‌ی امروزی کارایی بالایی داشته باشد را می‌توان در شبکه‌های اطلاعات-محور به‌کار گرفت. چرا که محدودیت کمتری در این شبکه‌ها وجود دارد \cite{ndn}.

\lr{CCN} از الگوریتم مسیریابی \lr{BGP}
با هماهنگ‌سازی آن برای فهمیدن نام و \lr{DONA} نیز از همین الگوریتم اما کمی متفاوت استفاده می‌کنند.
\lr{PURSUIT} هم روش مسیریابی خودش را به‌کار گرفته‌است.


\subsection{"چه" به جای "کجا"}

در شبکه‌ی امروزی دانستن مکان یک محتوا برای درخواست آن ضروری است به طوری‌که بسته‌های ارسالی در شبکه همگی دارای آدرس \lr{IP} مقصد می‌باشند. همین امر چالش‌هایی در زمینه‌ی تحرک\footnote{\lr{Mobility}} در شبکه به‌وجود می‌آورد. در این حالت مسیریاب‌ها بدون اطلاع از محتوای درخواستی، تنها قصد رساندن آن به مقصد مشخص‌شده را دارند.

در شبکه‌های محتوا-محور مسیریاب‌ها اطلاعات بیشتری درباره‌ی محتوای درخواست‌شده دارند. بررسی درخواست‌ها در لایه‌ی شبکه، این امکان را به مسیریاب‌ها می‌دهد که از محتوای بسته‌ی درخواستی با‌خبر شوند و در صورت امکان به آن پاسخ داده و یا برای مقصد بسته تصمیم‌گیری کنند(چیزی که در شبکه‌ی امروزی انجام نمی‌شود).

بنابراین در شبکه‌های محتوا-محور مکان محتوا اهمیتی نداشته و خود محتوا اهمیت پیدا می‌کند. همین امر به نهان‌سازی در این شبکه‌ها کمک شایان ذکری می‌کند. چرا که بدون توجه به مکان می‌توان محتواها را نهان کرد.


\section{معماری‌های شبکه‌های اطلاعات محور}
مثل شبکه‌های بی‌سیم ویا شبکه‌های اقتضایی\footnote{Ad-hoc Networks} عبارت شبکه‌های اطلاعات محور به تنهایی یک مفهوم عام بوده و در اصل شبکه‌ی خاصی تحت این عنوان وجود ندارد. در حال حاضر پروژه‌های مختلف اطلاعات محوری در حال توسعه بوده که در ادامه به چند مورد اشاره می‌شود و در نهایت شبکه‌ی محتوا محور، دقیق‌تر مورد بررسی قرار می‌گیرد.


\begin{itemize}
	\item
	\textbf{\lr{DONA\footnote{\lr{Data Oriented Network Architecture}}}}\cite{dona}: طراحی مجدد نام‌گذاری در اینترنت موجود در سال ۲۰۰۷ به صورت ناوابسته به معماری کنونی\footnote{\lr{Clean-slate}} 
	
	\item
	\textbf{\lr{CCN\footnote{\lr{Content Centric Network}}}}\cite{ndn}: طراحی صورت گرفته در سال ۲۰۰۹ با نگاهی به \lr{IP} که در قالب پروژه‌ی \lr{NDN}\footnote{\lr{Named Data Networking}} در حال توسعه است.
	
	\item
	\textbf{\lr{PSIRP\footnote{\lr{Publish-Subscribe Internet Routing Paradigm}}}}\cite{pursuit}: هم اکنون در قالب پروژه‌ی \lr{PURSUIT} در حال ادامه است.
	
	\item
	.................
	
	
\end{itemize}


CCN‌از نهان سازی همه چیز در همه جا استفاده می‌کند و مکانیزم جایگزینی نیز LRU است
