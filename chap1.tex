% Chapter 1
\chapter{مقدمه}

اینترنت امروزی پر از خدماتیست که هر یک به مرور و برای برآورده‌کردن یک نیاز خاص به آن اضافه شده‌اند. این افزایش تدریجی در کنار اینکه شبکه‌ی ساده‌ی اولیه را تبدیل به شبکه‌ای پیچیده و البته با قدرت بیشتر کرده، سربار‌ زیادی را بدان تحمیل می‌کند. این سربار با افزایش استفاده از ویدئو و به‌طور کلی محتوا در شبکه بیش از پیش به چشم می‌آید. در شبکه‌های اطلاعات-محور\footnote{\lr{Information Centric Networks (ICN)}} ایده‌ی اصلی حرکت از یک شبکه‌ی میزبان-مبنا\footnote{\lr{Host-Oriented}} به سمت یک شبکه‌ی محتوا-مبنا\footnote{\lr{Content-Oriented}} است. در این شبکه‌ها \textit{نام محتوا} نقش اصلی را بازی خواهد کرد و به دنبال آن اهمیت مکان(\lr{Where}) یک محتوا جای خود را به خود(\lr{What}) محتوا می‌دهد. در ادامه ....................


\section{چرا شبکه‌های اطلاعات-محور؟}
شبکه‌ی پیشنهاد شده‌ی اولیه تنها برای ارتباطات نقطه-به-نقطه\footnote{\lr{Point-to-point}}ی ساده و کاربردهای خاص طراحی شده بود. با فراگیر شدن آن، کاربری‌های جدیدی نیز بدان افزوده شد. صفحات وب و محتواهایی از قبیل صدا و تصویر شبکه را ملزم به تغییر کردند. تا جایی که امروزه خدماتی از جمله خدمات نام دامنه\footnote{\lr{Domain Name Service}}، شبکه‌های نظیر-به-نظیر\footnote{\lr{P2P Networks}}، شبکه‌های توزیع محتوا\footnote{\lr{Content Delivery Networks (CDN)}} و خدمات فراوان دیگری به صورت پوشش\footnote{\lr{Overlay}} روی شبکه استفاده می‌شوند(حتی امنیت هم امروزه به صورت پوششی رو شبکه است). در عین حال به عضوی جداناپذیر از شبکه تبدیل شده‌اند. برای درک بهتر نیاز به شبکه‌های اطلاعات-محور، توجه به مثال زیر خالی از لطف نیست.

برنامه‌نویسان در صورت نیاز به اعمال تغییرات گسترده در کدی که سال‌های قبل نوشته‌شده، تمایل به بازنویسی کد دارند. چرا که اجزای کد  دراین حالت هماهنگی بیشتری خواهند داشت. همچنین شرکت‌های نرم‌افزاری پس از چند سال استفاده و تغییر یک کد، آن‌را با نگاهی به کد فعلی، بازنویسی می‌کنند. شبکه‌های اطلاعات-محور، برای چنین منظوری ارائه شده‌اند. این شبکه‌ها بازنگری روی شبکه‌ی امروزی برای هماهنگی بیشتر اجزا و کاهش سربارهای موجود است. در این حالت، خدمات گفته‌شده در ساختار اصلی قرار می‌گیرند.

بنابراین انعطاف‌پذیری در توسعه از جمله مواردی‌است که شبکه‌ی امروزی به سادگی پاسخ‌گوی آن نیست. از طرفی تغییر نیاز کاربران که از سمت درخواست چندین متن به سوی نیازهای چند رسانه‌ای\footnote{\lr{Multimedia}} رفته و از طرف دیگر سربار‌هایی که برای پاسخ به نیازهای جدید نیاز است،  لزوم طراحی یک معماری جدید برای شبکه را گوشزد می‌کند.

یکی از راه‌حل‌هایی که پاسخ به نیاز کاربران را ساده‌تر می‌کند، نهان‌سازی داده‌ها در شبکه است. در ادامه به بررسی اهمیت این مورد پرداخته می‌شود.


\section{اهمیت نهان‌سازی}
از گذشته تا کنون، همواره در شبکه تلاش برای ارائه‌ی خدماتی بهتر ادامه داشته و در اصل تغییرات شبکه به همین منظور رخ می‌دهند. خدمات بهتر را می‌توان در تاخیر\footnote{\lr{Latency}}، زمان پاسخ\footnote{\lr{Response Time}}، بار سرور\footnote{\lr{Server Load}}، ترافیک و ازدحام\footnote{\lr{Congestion}} کمتر دانست. به‌وجود آمدن شبکه‌های توزیع محتوا در اصل به همین منظور انجام گرفته. در این شبکه‌ها مخازن\footnote{\lr{Storage}} بزرگی در سرتاسر شبکه قرار گرفته که نهان‌سازی داده‌ها را برعهده دارند.

نهان‌سازی در شبکه، چندان ایده‌ی جدیدی نیست. به طور مثال \cite{adaptive_web_caching} که در سال ۱۹۹۷ به بحث دراین باره پرداخته‌ و در گذر زمان به اهمیت آن افزوده شده، تا جایی که یکی از مهم‌ترین ویژگی‌ها در شبکه‌های اطلاعات-محور به شمار می‌رود.



در ادامه در فصل ........... به بررسی دقیق‌تر شبکه‌های اطلاعات-محور و در فصل ..................


